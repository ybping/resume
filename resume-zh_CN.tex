% !TEX TS-program = xelatex
% !TEX encoding = UTF-8 Unicode
% !Mode:: "TeX:UTF-8"

\documentclass{resume}
\usepackage{zh_CN-Adobefonts_external} % Simplified Chinese Support using external fonts (./fonts/zh_CN-Adobe/)
%\usepackage{zh_CN-Adobefonts_internal} % Simplified Chinese Support using system fonts
\usepackage{linespacing_fix} % disable extra space before next section
\usepackage{cite}


\begin{document}
\pagenumbering{gobble} % suppress displaying page number

\name{平亚斌}

\basicInfo{
  \email{yabping@gmail.com} \textperiodcentered\ 
  \phone{(+86) 159-6818-3014} \textperiodcentered\ 
  \linkedin[Yabin Ping]{https://www.linkedin.com/in/ping-yabin-a3225068?trk=nav_responsive_tab_profile}}
 
\section{\faGraduationCap\  教育背景}
\datedsubsection{\textbf{上海交通大学}, 上海, 上海}{2012.9 -- 2015.4}
\textit{硕士}\ 计算机科学与技术
\datedsubsection{\textbf{湖南农业大学}, 湖南, 长沙}{2007.9 -- 2011.6}
\textit{学士}\ 计算机科学与技术

\section{\faCogs\ IT 技能}
% increase linespacing [parsep=0.5ex]
\begin{itemize}[parsep=0.5ex]
	\item 编程语言: 熟悉Python、Golang、C/C++。
	\item 技能: 常用算法、Thrift、SOA、RabbitMQ、Celery、Redis、MySQL、PostgreSQL。
	\item 课程: 操作系统、数据结构、计算机网络、数据库原理。
\end{itemize}


\section{\faUsers\ 实习/项目经历}
\datedsubsection{\textbf{饿了么} 上海}{2015年5月 -- 至今}
\role{物流系统搭建}{开发经理}
\begin{onehalfspacing}
	物流配送系统对打通上下游的关系、提升用户体验有着至关重要的的关系;目前该系统属于饿了么的一个核心平台。我参与的主要工作包括:
\begin{itemize}
  \item 分析业务架构进行领域拆分,设计应用架构和技术架构; 并主导实现了整个系统平台。
  \item 基于soa架构实现各个服务,有服务注册、服务发现、服务管理、服务监控等功能。
  \item 针对物流最后一公里的配送设计调度算法、LBS服务提供配送效率。
  \item 目前正在做一些大数据和机器学习方面的工作。
\end{itemize}
\end{onehalfspacing}

\datedsubsection{\textbf{微软}}{2014年4月 -- 2014年7月}
\role{Azure计费管理模块}{实习生}
\begin{onehalfspacing}
该模块是一个网页功能模块,用来管理Azure云平台的一些计费功能,我参与的主要工作包括:
\begin{itemize}
  \item 基于SignalR技术,实现了服务器到客户端的推送,实时更新服务端的信息。
  \item 解决了一些安全问题, 主要包括XSS、CSRF、SQL注入攻击等。
\end{itemize}
\end{onehalfspacing}

\datedsubsection{\textbf{ 英特尔}}{2013 年7月 -- 2014年3月}
\role{DPDK VirtIO}{实习生}
\begin{onehalfspacing}
DPDK是一个是一组快速处理数据包的开发平台及接口, 我参与的主要工作包括:
\begin{itemize}
  \item 研究DPDK的两个功能VirtIO和IVSHMEM;
  \item 编写了一个自动化测试框架,实现自动测试基于母机、虚机的virtio功能。
\end{itemize}
\end{onehalfspacing}

% Reference Test
%\datedsubsection{\textbf{Paper Title\cite{zaharia2012resilient}}}{May. 2015}
%An xxx optimized for xxx\cite{verma2015large}
%\begin{itemize}
%  \item main contribution
%\end{itemize}

\section{\faHeartO\ 获奖情况}
\datedline{\textit{上海交通大学二等奖学金}}{2013年9月}
\datedline{IEEE Globecom最佳论文提名奖}{2014年12月}

\section{\faInfo\ 其他}
% increase linespacing [parsep=0.5ex]
\begin{itemize}[parsep=0.5ex]
  \item GitHub: https://github.com/yabping
\end{itemize}

%% Reference
%\newpage
%\bibliographystyle{IEEETran}
%\bibliography{mycite}
\end{document}

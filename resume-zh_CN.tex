% !TEX TS-program = xelatex
% !TEX encoding = UTF-8 Unicode
% !Mode:: "TeX:UTF-8"

\documentclass{resume}
\usepackage{zh_CN-Adobefonts_external} % Simplified Chinese Support using external fonts (./fonts/zh_CN-Adobe/)
%\usepackage{zh_CN-Adobefonts_internal} % Simplified Chinese Support using system fonts
\usepackage{linespacing_fix} % disable extra space before next section
\usepackage{cite}


\begin{document}
\pagenumbering{gobble} % suppress displaying page number

\name{平亚斌}

\basicInfo{
  \email{yabping@gmail.com} \textperiodcentered\ 
  \phone{(+86) 159-6818-3014} \textperiodcentered\ 
  \linkedin[Yabin Ping]{https://www.linkedin.com/in/ping-yabin-a3225068}}
 
\section{\faGraduationCap\  教育背景}
\datedsubsection{\textbf{上海交通大学}, 上海, 上海}{2012.9 -- 2015.4}
\textit{硕士}\ 计算机科学与技术
\datedsubsection{\textbf{湖南农业大学}, 湖南, 长沙}{2007.9 -- 2011.6}
\textit{学士}\ 计算机科学与技术

\section{\faCogs\ IT 技能}
% increase linespacing [parsep=0.5ex]
\begin{itemize}[parsep=0.5ex]
	\item 编程语言: 熟悉Python、Golang、C/C++、 C\#
	\item 技能: Algorithms、Machine Learning, Thrift、SOA、RabbitMQ、Celery、Redis、MySQL、PostgreSQL
	\item 课程: 算法设计与分析、基础代数、数论、密码学、操作系统、数据结构、计算机网络、数据库原理等
\end{itemize}


\section{\faUsers\ 实习/项目经历}

\datedsubsection{\textbf{微软}}{2017年5月 -- 至今}
\role{Azure Infrustructure Management Group}{高级软件开发工程师}
\begin{onehalfspacing}
	目前主要涉及的工作是AzureStack基础平台及其Service的监控平台AzureMonitor,包括Agent采集数据客户端, MDM(multi-dimension metrics)数据存储服务, GenevaRP查询服务。我参与的主要工作包括:
	\begin{itemize}
		\item 开发Agent metrics extension用于采集目标维度的时序数据;
		\item 优化复杂multi-dimension数据的聚合查询, 优化时序数据的压存储、利用bucket方案解决多维度数据的数据倾斜问题等。
		\item 开发Grafana Azure Monitor plugin,使得Grafana在为Azure Stack 提供支持。
		\item 分析服务间的Threat Model, 解决了一些安全问题, 主要包括认证、授权,XSS、CSRF、SQL注入攻击等;
	\end{itemize}
\end{onehalfspacing}

\datedsubsection{\textbf{饿了么} 上海}{2015年5月 -- 2017-5}
\role{物流系统搭建}{研发经理}
\begin{onehalfspacing}
	物流配送平台属于饿了么的一个核心平台,我参与的主要工作包括:
\begin{itemize}
 \item 参与分析业务架构进行领域拆分,设计应用架构和技术架构, 并主导实现了整个系统平台;
  \item 基于微服务架构实现各个核心服务。\\1.  各个服务启动后自动向zookepper注册,客户端通过watch相应的服务端地址, 通过rpc通信 \\2. 为了服务健壮,会有服务端的心跳检测、限流、以及熔断等异常处理机制
  \\3.  利用RabbitMQ主要用来解耦和消峰, Redis作为缓存,减少数据库的压力
  \\4.  数据库MySQL做了分库分表来解决单库单表的连接数、读写压力等性能瓶颈问题
  \item 针对物流最后一公里的配送设计调度算法, 这里主要用是设计多约束条件下的最优化问题以及对离线数据的模型训练,比如出餐时间预估、到店时间预估、订单-配送员压力平衡预估等模型;
  \\ 1.  主要设计的技术有数据预处理、模型对比选择、超参数调优、初始化权重设计等
  \item 针对在线调度算法主要研究开发了模拟退火算法、遗传算法,主要难点是超参数和遗传算子的设计;
  \item 目前在研究一些Deep Learning和Reinforcement Learning方面的技术;
\end{itemize}
\end{onehalfspacing}

\datedsubsection{\textbf{ 英特尔}}{2013 年7月 -- 2014年3月}
\role{DPDK VirtIO}{实习生}
\begin{onehalfspacing}
DPDK是一个是一组快速处理数据包的开发平台及接口, 我参与的主要工作包括:
\begin{itemize}
  \item 研究DPDK的两个功能VirtIO和IVSHMEM;
  \item 编写了一个自动化测试框架,实现自动测试基于母机、虚机的virtio功能。
\end{itemize}
\end{onehalfspacing}

% Reference Test
%\datedsubsection{\textbf{Paper Title\cite{zaharia2012resilient}}}{May. 2015}
%An xxx optimized for xxx\cite{verma2015large}
%\begin{itemize}
%  \item main contribution
%\end{itemize}

\section{\faHeartO\ 获奖情况}
\datedline{\textit{上海交通大学二等奖学金}}{2013年9月}
\datedline{IEEE Globecom最佳论文提名奖}{2014年12月}

\section{\faInfo\ 其他}
% increase linespacing [parsep=0.5ex]
\begin{itemize}[parsep=0.5ex]
  \item GitHub: https://github.com/yabping
\end{itemize}

%% Reference
%\newpage
%\bibliographystyle{IEEETran}
%\bibliography{mycite}
\end{document}
